\documentclass{article}
\usepackage[utf8]{inputenc}
\usepackage[spanish]{babel}
\usepackage{listings}
\usepackage{graphicx}
\graphicspath{ {images/} }
\usepackage{cite}

\begin{document}

\begin{titlepage}
    \begin{center}
        \vspace*{1cm}
            
        \Huge
        \textbf{Taller memoria}
            
        \vspace{0.5cm}
        \LARGE
        Informática 2
            
        \vspace{1.5cm}
            
        \textbf{Julián Mauricio Sánchez Ceballos}
            
        \vfill
            
        \vspace{0.8cm}
            
        \Large
        Despartamento de Ingeniería Electrónica y Telecomunicaciones\\
        Universidad de Antioquia\\
        Medellín\\
        Septiembre de 2020
            
    \end{center}
\end{titlepage}

\tableofcontents

\section{Sección introductoria}
Esta es la primera sección, podemos agregar algunos elementos adicionales y todo será escrito correctamente. Más aún, si una palabra es demasiado larga y tiene que ser truncada, babel tratará de truncarla correctamente dependiendo del idioma.

\section{Sección de contenido} \label{contenido}

Esta sección es para ver qué pasa con los comandos 
que definen texto

El paquete también agrega un comportamiento especial 
a <<estas marcas para hacer citas textuales>> tal como 
lo indican las reglas de la RAE. \cite{dirac}

\begin{lstlisting}
#include <stdio.h>
#define N 10
/* Block
 * comment */

\documentclass{article}
\usepackage[utf8]{inputenc}
\usepackage[spanish]{babel}
\usepackage{listings}
\usepackage{graphicx}
\graphicspath{ {images/} }
\usepackage{cite}

\begin{document}

\begin{titlepage}
    \begin{center}
        \vspace*{1cm}
            
        \Huge
        \textbf{Taller memoria}
            
        \vspace{0.5cm}
        \LARGE
        Informática 2
            
        \vspace{1.5cm}
            
        \textbf{Julián Mauricio Sánchez Ceballos}
            
        \vfill
            
        \vspace{0.8cm}
            
        \Large
        Despartamento de Ingeniería Electrónica y Telecomunicaciones\\
        Universidad de Antioquia\\
        Medellín\\
        Septiembre de 2020
            
    \end{center}
\end{titlepage}



\section{Taller} \label{taller}

    \subsection{Defina que es la memoria del computador}
    
    \subsection{mencione los tipos de memoria que conoce y haga una pequeña descripción de cada tipo}
    
    \subsection{describa la manera como se gestiona la memoria en un computador}
    
    \subsection{¿Qué hace que una memoria sea más rápida que otra? ¿Por qué esto es importante?}



\section{Conclusión} \label{conclulsion}

\bibliographystyle{IEEEtran}
\bibliography{references}

\end{document}
